\begin{table}[!h]
\centering
\caption{Cumulative effects (ATT) of COVID-19 Infection on Opposition to COVID-19 Policies}
\centering
\resizebox{\ifdim\width>\linewidth\linewidth\else\width\fi}{!}{
\begin{tabular}[t]{lcccc}
\toprule
  & IVHS & Log & Covid Cases & w/ Cases and Deaths\\
\midrule
ATT.avg & -0.327 & -0.327 & -0.329 & -0.327\\
S.E. & 0.117 & 0.117 & 0.118 & 0.117\\
CI.lower & -0.557 & -0.557 & -0.559 & -0.557\\
CI.upper & -0.097 & -0.097 & -0.098 & -0.098\\
p.value & 0.005 & 0.005 & 0.005 & 0.005\\
\addlinespace
Observations & 36722.000 & 36722.000 & 36722.000 & 36722.000\\
\bottomrule
\multicolumn{5}{l}{\rule{0pt}{1em}\textit{Note: }}\\
\multicolumn{5}{l}{\rule{0pt}{1em}Standard errors are presented in parentheses. All results presented use matrix completion methods and are estimated using the FECT library in R (Liu, Wang, Xu 2022). Models 1 and 2 use an inverse hyperbolic sine transformation and log+.1 transformation, respectively. Model 3 includes the number of COVID-19 cases per day in each legislator's constituency state. Model 4 includes the number of new cases and new deaths in each legislator's constituency state.}\\
\end{tabular}}
\end{table}
